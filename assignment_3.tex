%%%%%%%%%%%%%%%%%%%%%%%%%%%%%%%%%%%%%%%%%
% Short Sectioned Assignment
% LaTeX Template
% Version 1.0 (5/5/12)
%
% This template has been downloaded from:
% http://www.LaTeXTemplates.com
%
% Original author:
% Frits Wenneker (http://www.howtotex.com)
%
% License:
% CC BY-NC-SA 3.0 (http://creativecommons.org/licenses/by-nc-sa/3.0/)
%
%%%%%%%%%%%%%%%%%%%%%%%%%%%%%%%%%%%%%%%%%

%----------------------------------------------------------------------------------------
%	PACKAGES AND OTHER DOCUMENT CONFIGURATIONS
%----------------------------------------------------------------------------------------

\documentclass[paper=a4, fontsize=11pt]{scrartcl} % A4 paper and 11pt font size

\usepackage[T1]{fontenc} % Use 8-bit encoding that has 256 glyphs
\usepackage{fourier} % Use the Adobe Utopia font for the document - comment this line to return to the LaTeX default
\usepackage[english]{babel} % English language/hyphenation
\usepackage{amsmath,amsfonts,amsthm} % Math packages

\usepackage{lipsum} % Used for inserting dummy 'Lorem ipsum' text into the template

\usepackage{sectsty} % Allows customizing section commands
\allsectionsfont{\centering \normalfont\scshape} % Make all sections centered, the default font and small caps

\usepackage{fancyhdr} % Custom headers and footers
\pagestyle{fancyplain} % Makes all pages in the document conform to the custom headers and footers
\fancyhead{} % No page header - if you want one, create it in the same way as the footers below
\fancyfoot[L]{} % Empty left footer
\fancyfoot[C]{} % Empty center footer
\fancyfoot[R]{\thepage} % Page numbering for right footer
\renewcommand{\headrulewidth}{0pt} % Remove header underlines
\renewcommand{\footrulewidth}{0pt} % Remove footer underlines
\setlength{\headheight}{13.6pt} % Customize the height of the header

\numberwithin{equation}{section} % Number equations within sections (i.e. 1.1, 1.2, 2.1, 2.2 instead of 1, 2, 3, 4)
\numberwithin{figure}{section} % Number figures within sections (i.e. 1.1, 1.2, 2.1, 2.2 instead of 1, 2, 3, 4)
\numberwithin{table}{section} % Number tables within sections (i.e. 1.1, 1.2, 2.1, 2.2 instead of 1, 2, 3, 4)

\setlength\parindent{0pt} % Removes all indentation from paragraphs - comment this line for an assignment with lots of text

\newcommand{\matxx}[2] {
\begin{bmatrix}
  #1 \\
  #2 \\
\end{bmatrix}
}

\newcommand{\matxxx}[3] {
\begin{bmatrix}
  #1 \\
  #2 \\
  #3 \\
\end{bmatrix}
}

\newcommand{\matxxxx}[4] {
\begin{bmatrix}
  #1 \\
  #2 \\
  #3 \\
  #4 \\
\end{bmatrix}
}
\newcommand{\matxxxxx}[5] {
\begin{bmatrix}
  #1 \\
  #2 \\
  #3 \\
  #4 \\
  #5 \\
\end{bmatrix}
}

\newcommand{\matxxxxxx}[6] {
\begin{bmatrix}
  #1 \\
  #2 \\
  #3 \\
  #4 \\
  #5 \\
  #6 \\
\end{bmatrix}
}

\newcommand{\arrow}[1] {\xrightarrow[]{\text{#1}}}

%----------------------------------------------------------------------------------------
%	TITLE SECTION
%----------------------------------------------------------------------------------------

\newcommand{\horrule}[1]{\rule{\linewidth}{#1}} % Create horizontal rule command with 1 argument of height

\title{
\normalfont \normalsize
\textsc{Linear Algebra} \\ [25pt] % Your university, school and/or department name(s)
\horrule{0.5pt} \\[0.4cm] % Thin top horizontal rule
\huge A relationship between Linear Algebra and Machine Learning \\
\horrule{2pt} \\[0.5cm] % Thick bottom horizontal rule
}

\author{Yong Hoon Do, Sojung Kim} % Your name

\date{\normalsize\today} % Today's date or a custom date

\begin{document}

\maketitle % Print the title

%----------------------------------------------------------------------------------------
%	Machine Learning
%----------------------------------------------------------------------------------------
\section{What is Machine Learning}

Machine learning is defined as a type of artificial intelligence (AI) that provides computers
with the ability to learn without being explicitly programmed.
The term, Machine Learning may sound strange, but it is not an exaggeration to say that
we have been taking advantage of it for our entire Google searching experience
and also the experience purchasing items on Amazon, or selecting movies in Netflix.
Machine learning is used at almost every part of the stack at major search engines, and
anything that requires some sort of intelligence is often solved using machine learning.
Spelling suggestion/correction and also search ranking system are great examples that can be solved by Machine Learning. \\

A recommender system is one of the successful field that the concept of Machine Learning is deeply engaged with.
An algorithm called, \textbf{Collaborative Filtering} is one that commonly used nowadays,
and it works by building a database of preferences for items by a group of users.
For example, a new user, Guerrero, is matched against the database to discover neighbors,
which are other users who have historically had similar taste to Guerrero.
In this specific example, the solution may be defined as providing
a list of the best items to the user, so that the user, Guerrero could highly consider buying the recommended list of items.
We all agree that increasing a probability of purchasing items should give a benefit to the company, and
we can achieve this goal by using Machine Learning. \\

Since Machine learning is deeply relating with numerous properties of Linear Algebra,
we can also say that recommending the best items can be accomplished by utilizing many concepts of Linear Algebra.

%------------------------------------------------
\subsection{A relationship between Linear Algebra and Movie Recommender System}

Now, we want to show that Linear Algebra is highly engaging in Machine Learning by showing many properties of Linear Algebra
are really applied to a movie recommender system
since a movie recommender system is one of the successful application of Machine Learning.
Before we do jump into the specific properties of Linear Algebra how it is used in a recommender system,
let us consider a problem what we really want to achieve from doing it.

\bigskip

Netflix completely relies on Machine Learning to build a movie recommender system, and
the recommendations that come out from this system as an output drive \(60\%\) of Netflix's DVD rentals.
We can think of improving the accuracy of predictions about
how much someone is going to love a movie based on their movie preferences.
We would like to introduce an algorithm, \textbf{Collaborative Filtering} to discover
patterns in observed preference behavior (e.g. purchase history, item ratings and clikc counts) across
community users, and to predict new preferences based on those patterns. \\

We can concretize this problem using tools that we have learned in class.
Before we do analyze how the basic concepts of \textit{Linear Algebra} are used in this application,
let us define the problem in terms of mathematics. \\

Let us denote

\begin{enumerate}
	\item \(u\) is a set of users
	\[
		u \in \left\{ 1, \dots, U \right\}
	\]
	\item \(i\) is a set of items
	\[
		i \in \left\{ 1, \dots, M \right\}
	\]
	\item let \(T\) be a training set with observed,
	preferences in terms of \(\mathbb{R}\) as \(r_{ui}\) for some user-item pairs such that \((u,i)\).
	\[
		r_{ui} = \mbox{e.g. purchase indicator, item rating, click count, \dots}
	\]

	Our goal is that predicting unobserved preferences, or testing set \(Q\) with pairs such that \[(u,i) \not\subset T\]
	Based on those definitions, all we have to do is that filling numbers in unknown entries of sparse preference matrix.
	The rows indicate a number of users and columns indicate a number of items.
	\[
		\mathbf{R} =
		\matxxx
		{? & ? & 1 & \dots & 4}
		{3 & ? & ? & \dots & ?}
		{? & 5 & ? & \dots & 5}
	\]

	Since the real-valued unknown ratings are correlated with other pair of item-users,
	we can make solutions to this problem by using the ideas that we have learned in the class.
	(Notice that we are really seeing a matrix that defines a relationship between users and items!)

	As you may have already noticed, the problem is already defined as in terms of many ideas in Linear Algebra since
	users and items are really a vector space respectively, and they are having a linear relation.
	Though this is an explicit relationship between a recommender system and Linear Algebra,
	The following pages will show more insights that how those two complicated terms are correlated.

\end{enumerate}


\section{Set}
By the definition, set means "a set is a group of elements determined by one condition. Elements should be able to identify whether an element is contained in the collection or not, and whether two elements from the set are equal or not equal to each other."


In Recommendation system, the set could be the group of movie genre, searching key word, the rate of purchase, etc. For example, the set could be described as:
User A
\(
\begin{bmatrix}
5\\5\\?\\0\\0
\end{bmatrix}
\) User B \(
\begin{bmatrix}
5\\?\\4\\0\\0
\end{bmatrix}
\) User C \(
\begin{bmatrix}
0\\?\\0\\5\\5\\
\end{bmatrix}
\) User D  \(
\begin{bmatrix}
0\\0\\?\\4\\?
\end{bmatrix}
\)

\bigskip
In recommendation system, the feature set is necessary. The feature set is the set which is used as “input”. For example, if we make two feature sets: (Romance), (Action), they show how much ratio occupy for each movie.

Feature Set A (action) :
\(
\begin{bmatrix}
0.9\\1.0\\0.99\\0.1\\0
\end{bmatrix}
\) Feature Set B (romance):
\(
\begin{bmatrix}
0\\0.01\\0\\1.0\\0.9
\end{bmatrix}
\)

%--------------------------------------------------------------------------------
\bigskip
\section{Matrix}
By the Definition, matrix means "A matrix is an array of numbers or expressions arranged in rectangular shape. In a matrix, the horizontal line is called as a row, and vertical line is called as a column."


\bigskip
In Recommendation system, column and row could be expressed as users and products or users and rating. For example, we can make the matrix with the sets which are above.
\[
\begin{bmatrix}
Movie & A & B & C & D & (Romance) & (action) \\
Movie1 & 5 & 5 & 0 & 0 & 0.9 & 0 \\
Movie2 & 5 & ? & ? & 0 & 1.0 & 0.01\\
Movie3 & ? & 4 & 0 & ? & 0.99 & 0 \\
Movie4 & 0 & 0 & 5 & 4 & 0.1 & 1.0 \\
Movie5 & 0 & 0 & 5 & ? & 0 & 0.9
\end{bmatrix}
\]
%--------------------------------------------------------------------------------
\bigskip
\section{Transpose}

By the Definition, Transpose is that the element at row and column in the original matrix is placed at row and column of the transpose.



The reason why transpose is necessary is because, in multiplication of matrix, the column of one matrix and the row of another matrix should be the same. Thus, the transpose is necessary.



For example, when we estimate the undecided point by user, we need to calculate it. When we calculate it, if we do not transpose the matrix, it will be (1*3) * ( 1*3). However, this multiplication cannot be possible by Theorem of Multiplication. So, we need to do transpose it in order to make (1*3) * (3 *1).
\(
x^{movie 3} =
\begin{bmatrix}
1\\0.99\\0\\
\end{bmatrix}
\theta ^{A} = \begin{bmatrix}
0 \\ 5 \\ 0
\end{bmatrix}
\)

For finding the estimate rating for movie 3
\(x^{movie3} \theta ^{A^T}\)
\[
=
\begin{bmatrix}
1\\0.99\\0\\
\end{bmatrix}
*
\begin{bmatrix}
0 & 5 & 0
\end{bmatrix}
=
0 + 4.95 + 0 = 4.95
\]

Therefore, we can estimate user A may give 4.95 points for movie 3.


%------------------------------------------------

\subsubsection{Heading on level 3 (subsubsection)}

\lipsum[3] % Dummy text

\paragraph{Heading on level 4 (paragraph)}

\lipsum[6] % Dummy text

%----------------------------------------------------------------------------------------
%	PROBLEM 2
%----------------------------------------------------------------------------------------

\section{Lists}

%------------------------------------------------

\subsection{Example of list (3*itemize)}
\begin{itemize}
	\item First item in a list
		\begin{itemize}
		\item First item in a list
			\begin{itemize}
			\item First item in a list
			\item Second item in a list
			\end{itemize}
		\item Second item in a list
		\end{itemize}
	\item Second item in a list
\end{itemize}

%------------------------------------------------

\subsection{Example of list (enumerate)}
\begin{enumerate}
\item First item in a list
\item Second item in a list
\item Third item in a list
\end{enumerate}

%----------------------------------------------------------------------------------------

\end{document}
