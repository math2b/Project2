%%%%%%%%%%%%%%%%%%%%%%%%%%%%%%%%%%%%%%%%%
% Short Sectioned Assignment
% LaTeX Template
% Version 1.0 (5/5/12)
%
% This template has been downloaded from:
% http://www.LaTeXTemplates.com
%
% Original author:
% Frits Wenneker (http://www.howtotex.com)
%
% License:
% CC BY-NC-SA 3.0 (http://creativecommons.org/licenses/by-nc-sa/3.0/)
%
%%%%%%%%%%%%%%%%%%%%%%%%%%%%%%%%%%%%%%%%%

%----------------------------------------------------------------------------------------
%	PACKAGES AND OTHER DOCUMENT CONFIGURATIONS
%----------------------------------------------------------------------------------------

\documentclass[paper=a4, fontsize=12pt]{scrartcl} % A4 paper and 11pt font size

\usepackage[T1]{fontenc} % Use 8-bit encoding that has 256 glyphs
\usepackage{fourier} % Use the Adobe Utopia font for the document - comment this line to return to the LaTeX default
\usepackage[english]{babel} % English language/hyphenation
\usepackage{amsmath,amsfonts,amsthm} % Math packages

\usepackage{lipsum} % Used for inserting dummy 'Lorem ipsum' text into the template

\usepackage{sectsty} % Allows customizing section commands
\allsectionsfont{\centering \normalfont\scshape} % Make all sections centered, the default font and small caps

\usepackage{fancyhdr} % Custom headers and footers
\pagestyle{fancyplain} % Makes all pages in the document conform to the custom headers and footers
\fancyhead{} % No page header - if you want one, create it in the same way as the footers below
\fancyfoot[L]{} % Empty left footer
\fancyfoot[C]{} % Empty center footer
\fancyfoot[R]{\thepage} % Page numbering for right footer
\renewcommand{\headrulewidth}{0pt} % Remove header underlines
\renewcommand{\footrulewidth}{0pt} % Remove footer underlines
\setlength{\headheight}{13.6pt} % Customize the height of the header
\linespread{1.5}
\numberwithin{equation}{section} % Number equations within sections (i.e. 1.1, 1.2, 2.1, 2.2 instead of 1, 2, 3, 4)
\numberwithin{figure}{section} % Number figures within sections (i.e. 1.1, 1.2, 2.1, 2.2 instead of 1, 2, 3, 4)
\numberwithin{table}{section} % Number tables within sections (i.e. 1.1, 1.2, 2.1, 2.2 instead of 1, 2, 3, 4)

\setlength\parindent{0pt} % Removes all indentation from paragraphs - comment this line for an assignment with lots of text

\newcommand{\matxx}[2] {
\begin{bmatrix}
  #1 \\
  #2 \\
\end{bmatrix}
}

\newcommand{\matxxx}[3] {
\begin{bmatrix}
  #1 \\
  #2 \\
  #3 \\
\end{bmatrix}
}

\newcommand{\matxxxx}[4] {
\begin{bmatrix}
  #1 \\
  #2 \\
  #3 \\
  #4 \\
\end{bmatrix}
}
\newcommand{\matxxxxx}[5] {
\begin{bmatrix}
  #1 \\
  #2 \\
  #3 \\
  #4 \\
  #5 \\
\end{bmatrix}
}

\newcommand{\matxxxxxx}[6] {
\begin{bmatrix}
  #1 \\
  #2 \\
  #3 \\
  #4 \\
  #5 \\
  #6 \\
\end{bmatrix}
}

\newcommand{\arrow}[1] {\xrightarrow[]{\text{#1}}}

%----------------------------------------------------------------------------------------
%	TITLE SECTION
%----------------------------------------------------------------------------------------

\newcommand{\horrule}[1]{\rule{\linewidth}{#1}} % Create horizontal rule command with 1 argument of height

\title{
\normalfont \normalsize
\textsc{Linear Algebra} \\ [25pt] % Your university, school and/or department name(s)
\horrule{0.5pt} \\[0.4cm] % Thin top horizontal rule
\huge A relationship between Linear Algebra and Machine Learning \\
\horrule{2pt} \\[0.5cm] % Thick bottom horizontal rule
}

\author{Yong Hoon Do, Sojung Kim} % Your name

\date{\normalsize\today} % Today's date or a custom date

\begin{document}

\maketitle % Print the title

%----------------------------------------------------------------------------------------
%	Machine Learning
%----------------------------------------------------------------------------------------
\section{What is Machine Learning}

Machine learning is defined as a type of artificial intelligence (AI) that
provides computers with the ability to learn without being explicitly
programmed. The term, Machine Learning may sound strange, but it is not an
exaggeration to say that we have been taking advantage of it for our entire
Google searching experience and also the experience purchasing items on Amazon,
or selecting movies in Netflix. Machine learning is used at almost every part of
the stack at major search engines, and anything that requires some sort of
intelligence is often solved using machine learning. Spelling
suggestion/correction and also search ranking system are great examples that can be solved by Machine Learning. \\

A recommender system is one of the successful field that the concept of Machine
Learning is deeply engaged with. An algorithm called,
\textbf{Collaborative Filtering} is one that commonly used nowadays, and it works by building a
database of preferences for items by a group of users. Fo  example, a new user,
Guerrero, is matched against the database to discover neighbors, which are other
users who have historically had similar taste to Guerero.
In this specific example, the solution may be defined as providing
a list of the best items to the user, so that the user, Guerrero could highly
consider buying the items that recommended by the system.
We all agree that increasing a probability of purchasing items should give a
benefit to the company, and we can achieve this goal by using Machine Learning. \\

Since Machine learning is deeply relating with numerous properties of Linear Algebra,
we can also say that recommending the best items can be accomplished by
utilizing many concepts of Linear Algebra. \\

In this paper, we would not address how the algorithm really works in detail,
but will discover how the basic concepts of Linear Algebra are
applied to one of the hottest software, a recommender system that is derived
from Machine Learning.

%------------------------------------------------
\section{A relationship between Linear Algebra and Movie Recommender System}

Now, we want to show that Linear Algebra is highly engaging in Machine Learning
by showing many properties of Linear Algebra are really  applied to a movie
recommender system as a foundation concepts. Before we do jump into the specific
properties of Linear Algebra how it is used in a recommender system,
let us consider a problem what we really want to achieve from doing it.

\bigskip

\subsection{What Does A Movie Recommender System Do For Users?}

Netflix completely relies on Machine Learning to build a movie recommender system, and
the recommendations that come out from this system as an output drive \(60\%\)
of Netflix's DVD rentals. We can think of improving the accuracy of predictions
how much someone is going to love a movie based on their movie preferences.
Now, Netflix uses a well-known algorithm, called \textbf{Collaborative Filtering} to discover
patterns in observed preference behavior (e.g. purchase history, item ratings and clikc counts) across
community users, and to predict new preferences based on those patterns. \\

We can concretize this problem using tools that we have learned in class.
Before we do analyze how the basic concepts of \textit{Linear Algebra} are used in this application,
let us define the problem in terms of mathematical tools that we should have seen
in class. \\

Let us denote

\begin{enumerate}
	\item \(X\) is a set of users
	\item \(S\) is a set of movies
  \item \(R\) is a set of ratings, that should be \(r_{ui}\) for some user-movie
    pairs such that \((u,i)\).
  \item \(U\) is a linear transformation that takes two vectors spaces \(X\) and
    \(S\) as an input, and generates a set of ratings \(R\).
  \end{enumerate}

	Our primary goal is that building a brand new set of predicted ratings for
  each user such that 
  \[
    U : X \times S \arrow{} R
  \]

  As users give more ratings on movies, more accurated ratings will be offered
  to users by the Collaborative Filtering Algorithm.

  \section{A Relationship between Linear Algebra and Collaborative Filtering}

  As we have already introduced an idea that basic concepts of Linear Algebra are
  indispensable foundation to have Collaborative Filtering Algorithm, we would
  like to analyze how these concepts are really used in the algorithm. We are
  going to look at a few key concepts of Linear Algebra to realize how they are
  deeply connected:
  \begin{enumerate}
  \item Linear Transformation
  \item Linear Combination
  \item Basis and Dimension
  \item Nonsingularity
  \end{enumerate}

  The other fundamental concepts like a set, a matrix, a vector space and other
  tools like a matrix multiplication and matrix transpose are also significant
  to discover the deep engagement with the algorithm, but we find that these
  concepts are so natrual when it comes to prove the relationship between Linear
  Algebra and the Collaborative Filtering Algorithm. They will be frequently
  appearing all over the section. \\

  \subsection{Linear Transformation}

  The first essential idea that makes a relationship between the algorithm and
  Linear Algebra should be Linear Transformation since it defines a function. To
  give predicted ratings for community users as an output, we need an input!

  As you may have already noticed, it is defined as
  \[
    U : X \times S \arrow{} R
  \]

  \(U\) is a linear transformation that takes \(X \times S\) as an input, which
  includes two vector spaces; one is a set of users and another one is a set of
  movies. Multiplying these two vectors allows to find out the third vector
  space, namely a set of ratings beloing to each users in a set of users. \\

  The type of ratings that we are using in this paper would be divided into two;
  one is a rating that a user granted in person, and another rating should be a
  predicted rating that granted by the Collaborative Filtering Algorithm. This
  brief defintion may bring us to realize that we may have a matrix with unknown
  ratings, which are eventually known as a natrual number like \([0,5]\), but
  they are up on the same matrix. We can look this matrix so that we can
  understand what the algorithm is really trying to accomplish. \\

	\[
		\mathbf{M} =
		\matxxx
		{? & ? & 1 & \dots & 4}
		{3 & ? & ? & \dots & ?}
		{? & 5 & ? & \dots & 5}
	\]

  Each row indicates a user, each column indicates a movie and each entry in the
  matrix indicates a rating that the user directly granted. Then, the entries
  filled in a question mark are the ratings that eventually the algorithm would
  find out, and it will be offered to users such that
  \[
    r_{\mathbf{x}} = \left\{5,5,5,4,3,1,\dots,1\right\}
  \]

  where \(r_x\) is the vector the user \(\mathbf{x}\)'s predicted ratings. This
  notion defines the problem simpler as all we have to do is that filling
  numbers in unknown entries of sparse matrix.

  % ----------------------------------------------------------------------------------------
  %	Conclusion
  % ----------------------------------------------------------------------------------------

  \section{Conclusion}

  Why Linear Algebra is a fundamental element in Machine Learning? How concepts of
  Linear Algebra are used for Recommender system?
  Basis, Linear Transformation, a set, a matrix?

  % ----------------------------------------------------------------------------------------

\end{document}
